\chapter*{Résumé}
\addcontentsline{toc}{chapter}{Résumé}
\medskip
        Dans le cadre de mon stage au sein du \dsc, j'ai développé trois applications visant à permettre à des versions allégées d'images 2D et objets 3D d'intégrer leurs métadonnées structurelles et descriptives au cadre international d'interopérabilité des images (\iiif~- IIIF). La question de recherche est la suivante : est-il réellement possible de garantir la \psdn archivistiques en utilisant IIIF ?\\
        
        Le principal défi aujourd'hui réside dans la gestion de données hétérogènes provenant d'objets intrinsèquement variés, ainsi que de différentes organisations ayant des objectifs distincts. L'enjeu est d'éviter l'effet \enquote{\eng{setting and forgetting}} en cherchant à rendre la description et l'exploitation des données pérennes entre diverses organisations. De plus, l'utilisation de ces données soulève des problèmes et des impacts environnementaux. Par conséquent, l'optimisation de la préservation numérique et de l'interopérabilité apparaît comme une priorité à développer, car elle permettrait une utilisation efficace des données dans un contexte patrimonial.\\

        \textbf{Mots-clés:} IIIF~; \psdn~; \gco~; gestion de données~; \py~; \html~; \css~; \JS~; \fe~.\\
	
    	\textbf{Informations bibliographiques~:} Gilmar Ballestrim, \textit{Vers une persistance numérique des collections}, mémoire de Master \g{Technologies numériques appliquées à l'histoire}, dir. Edward Gray, \enc, septembre 2024.
