\mychapter{Introduction}
\addcontentsline{toc}{chapter}{Introduction}

    Programmer, dans notre société contemporaine, ne se résume plus à écrire des lignes de code : c’est également une activité qui s’inscrit dans un contexte social, politique et environnemental complexe. L’essor du concept de \gco met en lumière la nécessité de prendre en compte les impacts environnementaux et sociaux de la production de logiciels. C'est pour cette raison que dans le cadre de mon stage au sein du \DSC, j'ai été amené à développer deux applications éco-responsables visant à intégrer des versions allégées d'images 2D et modèles 3D, ainsi que leurs métadonnées structurelles et descriptives, au sein du cadre international d'interopérabilité des images, connu sous le nom d'\iiif~(IIIF). Ce travail s'inscrit dans une problématique plus large, soulevée par la question de recherche suivante : est-il réellement possible de garantir la persistance des données archivistiques en utilisant IIIF ?\\

    Cette interrogation prend toute son importance dans le contexte actuel, où la gestion de données hétérogènes constitue un défi majeur. Ces données, issues d'objets intrinsèquement variés et de différentes organisations, posent des problèmes d'harmonisation et de pérennisation. Le défi réside notamment dans la nécessité d'éviter l'effet \enquote{\eng{setting and forgetting}}, c'est-à-dire l'abandon de données après leur archivage initial, en développant des solutions qui assurent une description et une exploitation durable des informations à travers diverses organisations. De plus, l'usage de ces données numériques soulève des questions environnementales, rendant l'optimisation de leur préservation et interopérabilité une priorité.\\

    En outre, programmer est aussi une activité qui s'inscrit dans un moment historique précis. Les codes conçus ont pour but la préservation patrimoniale : bien qu'ils aient une finalité spécifique, ils sont souvent crées dans les contextes collectifs et d'\opso et sont susceptibles de devenir une marchandise. Cette discussion sera approfondie, mais on peut dès à présent souligner que rares sont les questionnements concernant les effets de la codification et la manière dont elle masque les relations sociales qui impactent, directement ou indirectement, non seulement l’environnement, mais aussi la vie des individus. Autrement dit, la réflexion de cette étude n'est pas d'épuiser le sujet de la relation entre la programmation et l'environnement, puisque cela est un débat déjà connu, notre idée est plutôt de mettre en exergue le fait que ces deux dimensions sont forcément liées à un mode de reproduction historique de la vie. En effet, comme Karl Marx l’avait souligné, les idées ne surgissent pas dans le vide, mais sont enracinées dans la réalité matérielle \footnote{Pour Marx, les Jeunes-Hégéliens restaient enfermés dans une pure spéculation philosophique, sans chercher à relier leurs idées à la société concrète dans laquelle ils vivaient. C'est de là que vient la célèbre phrase de Marx \enquote{Ce que sont les individus dépend donc des conditions matérielles de leur production} \cite[p.~44-45]{marx1968}.}. Dans notre société, les moyens de production sont principalement capitalistes. Si le capitalisme n'est pas un système transhistorique, comme l'a également souligné Moishe Postone \footnote{A l'instar de Postone, la théorie du travail qui intéresse cette étude est celle historiquement spécifique à la société capitaliste, non une sorte de capitalisme qui s'étend à toutes les sociétés de forme transhistorique. C'est justement cette dernière notion qui représente la critique centrale du livre de l'auteur. En outre, cette théorie va de pair avec la notion qui implique que le capitalisme est l'une des caractéristiques de base de la modernité \cite[p.~9-21]{postone1996}.}, il imprègne une grande partie des relations sociales, dont la programmation est l'une des manifestations.\\
    
    Toutefois, ce travail tente de penser des possibilités pour restituer un autre mode de programmation dans cette configuration sociale. Or, l’histoire de la programmation évidemment n'est pas une simple histoire de la guerre ou du capitalisme. Les discussions sur l’utilisation des logiciels libres et \opso, à l'instar des \eng{Free Open Source Software} (FOSS), ont une histoire liée à la contre-culture aux États-Unis : ces aspirations suggéraient beaucoup plus l’idée d’une quête pour la liberté des individus que de répondre aux aspirations capitalistes \footnote{Voir \cite{shi2014mainstreaming}.}.\\

    De plus, ce mémoire est également le moment de remettre en question certains concepts tels que le document, l'objet et l'artefact : ces définitions, qui font l'objet de débats terminologiques depuis longtemps, sont de plus en plus placées dans un contexte d'études interdisciplinaires. Un exemple de ce que nous explorerons et défendrons dans ce travail est le champ interdisciplinaire de la culture matérielle, qui nous intéresse de près, car ces définitions englobent des dimensions historiques qui influent sur les manières dont nous conservons les choses avec lesquelles nous interagissons et, par conséquent, sur la façon dont nous transformons ces choses en données ou comment nous concevons les données, puisque les données présentent ces deux aspects non négligeables : celui de choses qui sont devenues numériques ou celui de choses qui naissent numériques.\\
    
    Ce mémoire est structuré en trois parties principales, chacune visant à explorer en profondeur les différentes dimensions du travail effectué et à situer les développements réalisés dans leur contexte théorique et pratique.\\
    
    La première partie, intitulée \textbf{Production au sein du \dsc}, se concentre sur la mise en place du cadre institutionnel et technologique dans lequel s'inscrit ce projet. Cette section commence par un examen détaillé du \dsc, une institution suisse dédiée à la préservation numérique du patrimoine culturel. Le premier chapitre retrace l'histoire et la fondation du \dsc, en mettant l'accent sur ses liens avec l'initiative de l'\opdt en Suisse et sur les modèles de financement qui soutiennent ses activités. Ensuite, le contexte technique du IIIF est introduit. Ce cadre d'interopérabilité, crucial pour la gestion et le partage des images patrimoniales, est présenté à travers ses différentes versions, depuis l'API 3.0 jusqu'aux développements envisagés pour l'API 4.0. Enfin, cette partie aborde les concepts fondamentaux liés à la persistance des données numériques, en clarifiant les distinctions entre documents, objets et données, tout en soulignant les défis posés par la numérisation du patrimoine. Une attention particulière est mise sur les enjeux spécifiques du stage, les objectifs des \eng{pipelines} développés, et les limitations rencontrées dans l'automatisation des processus à travers l'utilisation de technologies comme \py, \textsc{flask}, \textsc{bootstrap}, et \textsc{docker}.\\
    
    La deuxième partie, dont le titre est \textbf{De l'image au standard IIIF : conceptions des \eng{pipelines} de traitement}, se consacre à l'aspect technique du projet, détaillant la conception et la mise en œuvre des \eng{pipelines} de traitement d'images pour leur intégration dans le standard IIIF. Cette section se divise en plusieurs chapitres qui expliquent, étape par étape, le processus d'optimisation des images et des objets numériques pour répondre aux exigences du IIIF. Elle explore notamment le développement des outils \cvt et \msh, qui ont été conçus pour faciliter cette intégration. Une attention particulière est accordée à la structure de ces outils, à leur architecture logicielle, ainsi qu'à l'expérience utilisateur. Finalement, la discussion porte sur les défis techniques et des solutions apportées pour enrichir les métadonnées et alimenter les manifestes IIIF lors de la conception d'\diiif.\\
    
    La troisième partie, formulée en tant qu'\textbf{Évaluation des outils et méthodes : bilan et perspectives}, propose une analyse critique des résultats obtenus au cours du développement des applications et aussi une réflexion à propos des outils de gestion du temps. Elle commence par une évaluation de la performance des outils créés, en examinant leur efficacité, leur robustesse et leur capacité à répondre aux objectifs initiaux du projet. Est discutée ensuite la façon dont les outils de gestion ont été intégrés lors du stage. Cette partie se poursuit avec une réflexion sur les perspectives d'évolution des outils, en identifiant des fonctionnalités supplémentaires qui pourraient être développées pour améliorer encore davantage l'intégration des images et métadonnées dans le cadre IIIF. Enfin cette section se termine sur des pistes de recherches futures, en considérant l'impact potentiel des technologies explorées sur la préservation du patrimoine culturel numérique.\\
    
    Ce mémoire vise ainsi à fournir une analyse des enjeux techniques et institutionnels liés à la gestion des données patrimoniales numériques, tout en proposant des solutions éco-resposables pour améliorer leur interopérabilité et leur préservation à travers l'utilisation du standard IIIF, sans oublier la dimension historique sous-jacente à cette reproduction historique de vie.