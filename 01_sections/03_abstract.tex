\chapter*{Abstract}
\addcontentsline{toc}{chapter}{Abstract}
\medskip
    During my internship at the \dsc, I developed three applications aimed at enabling lighter versions of 2D images and 3D objects to integrate their structural and descriptive metadata into the International Image Interoperability Framework (\iiif~- IIIF). The research question is as follows: Is it truly possible to ensure archival data persistence using IIIF?\\
    
    The main challenge today lies in managing heterogeneous data from intrinsically varied objects and different organizations with distinct objectives. The goal is to avoid the \enquote{\eng{setting and forgetting}} effect by striving to make data description and utilization sustainable across various organizations. Additionally, the use of this data raises environmental issues and impacts. Therefore, optimizing digital preservation and interoperability emerges as a priority to be developed, as it would allow for the efficient use of data in a heritage context.\\

\textbf{Keywords:} IIIF~; data persistence~; \gco~; data management~; \py~; \html~; \css~; \JS~; \fe~.\\

\textbf{Bibliographic Information:} Gilmar Ballestrim, \textit{Towards Digital Persistence of Collections}, Master's thesis in \g{Digital Technologies Applied to History}, dir. Edward Gray, \enc, September 2024.
