\mychapter{Conclusion}
\addcontentsline{toc}{chapter}{Conclusion}

Programmer, dans notre société contemporaine, ne se résume plus à écrire des lignes de code : c’est une activité qui s’inscrit dans un contexte social, politique et environnemental complexe. L’essor du concept de \gco met en lumière la nécessité de prendre en compte les impacts environnementaux et sociaux de la production de logiciels. L'IIIF, en tant que standard pour la manipulation et la présentation de ressources numériques, est souvent présenté comme une solution pour assurer la persistance des données.

L'IIIF, de par sa conception flexible et ouverte, offre indéniablement de perspectives pour garantir la persistance des ressources numériques. En tant que standard, il favorise l'interopérabilité entre différents systèmes et facilite ainsi la préservation à long terme des collections numériques, comme c'est le cas chez \dsc. Sa modularité permet aux institutions culturelles et aux chercheurs d'adapter les outils et les services à leurs besoins spécifiques, sans être contraints par des normes trop rigides.

Sur le plan technique, le stage a permis d'atteindre l'objectif fixé, à savoir créer de \eng{pipelines} automatisés en \py afin de gérer les images 2D et les modèles 3D. Les résultats obtenus sont très encourageants, démontrant une efficacité significative par rapport aux logiciels tels que \gmp et \mlb, tout en utilisant des méthodes similaires.

De plus, ce stage a ouvert de nouvelles perspectives de développement dans le domaine du patrimoine numérique. En particulier, la possibilité de créer \diiif, un code \py pour interagir avec les nouvelles APIs IIIF. Bien que cette solution soit encore en développement, elle offre de prometteuses perspectives d'association de fichiers \textsc{json} aux modèles 3D. Le défi consiste à lier les fichiers \textsc{.glb} et \textsc{.gltf} aux manifestes IIIF de manière intuitive, en créant une structure similaire à celle de \cvt et \msh, accessible à tous types d'utilisateurs, des novices aux experts. Cette approche s'inscrit dans une démarche de conception centrée utilisateur, visant à offrir une expérience utilisateur complète.

Les compétences en modélisation de données se sont révélées essentielles pour ce projet, chaque \textit{pipeline} nécessitant une bonne capacité d'abstraction et d'anticipation des problèmes. En revanche, bien que les connaissances en \textsc{sparql} soient utiles pour comprendre certains contextes, notamment ceux liés aux données \textsc{rdf}, elles n'ont pas été utilisées de manière intensive dans ce projet. Les problématiques liées à \textsc{sparql} étaient plutôt liées au développement de réponses pour les APIs IIIF \eng{Presentation} et \eng{Image}.

Il est important de noter que si \py a été choisi pour sa flexibilité, sa scalabilité et sa facilité d'utilisation, d'autres langages comme Rust et Go pourraient être envisagés pour certaines tâches spécifiques, notamment la gestion efficace de données.

Enfin, il serait intéressant de publier l'application développée sur le DockerHub du \dsc afin de la mettre à disposition de la communauté et de faciliter son utilisation par d'autres institutions confrontées à des problématiques similaires de gestion des images 2D et modèles 3D.

Sur le plan théorique, le \gco ne constitue pas seulement une démarche écologiquement responsable, il s’inscrit également dans les intérêts des grandes corporations qui, souvent, utilisent ce discours environnemental à des fins de \gwsh. Ce dernier obscurcit les véritables implications du développement technologique en réduisant la complexité des relations sociales à des questions individuelles de consommation consciente. En agissant ainsi, le \gwsh empêche une analyse critique des structures de pouvoir qui sous-tendent la production et la consommation technologiques.

Néanmoins, cette réflexion a enseigné qu’il est possible d’adopter d’autres manières de programmer, même si cette activité est ancrée dans une certaine configuration sociale qui vise souvent le profit. Il est intéressant d'intégrer les réflexions autour du \gco au processus de codage.

La réflexion sur les différentes conceptions des données révèle également un substrat social tout aussi crucial. Dans les sciences dites dures, la notion de données est purement technique, alors que dans les sciences humaines, le contexte de production des données est souvent d’un grand intérêt. Bien que cette notion dans les sciences humaines ouvre la discussion sur la nature des données en relation avec les archives, elle n’aborde pas nécessairement comment ces données sont utilisées et dans quels buts. Il est important de rappeler que les données peuvent, dès leur conception, être l’objet de conflits. Ces conceptions distinctes suggèrent que programmer est également un acte politique, et tout comme les organisations poursuivent différents objectifs avec les données, la manière dont nous codons est fondamentale.

Une autre réflexion suscitée par la rédaction de ce travail concerne la gestion des données, qui implique la manière dont nous utilisons nos ressources. En effet, l’utilisation de \textit{bits} correspond concrètement à l’utilisation d’un espace physique sur la planète. Ces données sont fréquemment stockées dans des centres de données. Dans ce contexte, notre pratique, qui pourrait sembler anodine, peut être perçue comme une pratique opérant un fétichisme dont nous devons nous rendre compte. 

Enfin, programmer dans une société capitaliste, sans réfléchir aux implications de nos actions, relève de la dialectique du discours technologique. Comme d’autres auteurs l’ont observé, même dans une perspective de discussion sur la liberté, il est difficile d’affirmer qu’il existe une manière totalement libre de programmer dans une société où certains présupposés sont naturalisés, typiques d’une société capitaliste. Nous avons tendance à voir la discussion environnementale et technologique comme distincte de la lutte pour l’émancipation sociale.

Programmer, tout comme produire n’importe quel autre bien, s’inscrit dans un système économique qui prône le profit et l’accumulation de capital. La recherche d’alternatives plus justes et durables exige une réflexion approfondie sur le rôle des programmeurs dans la construction d’un avenir plus équitable et écologique.

En ce sens, programmer, c’est aussi rêver d’une société différente, où les individus pourraient bénéficier non seulement des avantages de la programmation, mais aussi de la préservation de l’environnement. Il ne s’agit pas d’une perspective morale individualiste, où la responsabilité incombe à l’individu, mais plutôt d’une perspective collective de changement social, où nous utilisons les contradictions inhérentes au système capitaliste pour proposer une transformation ou, dans une vision plus utopique, son démantèlement.